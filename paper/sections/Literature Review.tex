
Traditional image denoising methods often rely on handcrafted filters or optimization algorithms, which may not generalize well to different noise types. Deep learning-based approaches have proven to be powerful tools for image denoising, achieving state-of-the-art performance on a variety of noise types.

One of the pioneering works in deep learning-based image denoising is DnCNN\cite{Zhang2017}, a deep convolutional neural network designed to remove Gaussian noise. DnCNN demonstrates the ability to handle blind Gaussian denoising, where the noise level is unknown, and can be extended to other image restoration tasks such as super-resolution. Other CNN-based methods have been proposed, including FFDNet\cite{Zhang2018}, which achieves fast and flexible denoising by incorporating noise level information into the network.

While CNNs excel at capturing local dependencies in images, they often struggle to model global image context, which can be crucial for high-quality image restoration. To address this limitation, Transformers, originally developed for natural language processing, have been adapted for image restoration tasks. Transformers utilize self-attention mechanisms to capture global dependencies between pixels, enabling them to effectively learn image priors and achieve impressive results on various restoration tasks.

One notable example is Restormer\cite{Zamir2022}, an efficient Transformer model designed for high-resolution image restoration. Restormer incorporates several key designs in its building blocks, such as multi-head attention and feed-forward networks, to capture long-range pixel dependencies while remaining computationally feasible for large images. Restormer has achieved state-of-the-art performance on various image restoration tasks, including deraining, deblurring, and denoising.

In addition to Gaussian noise, impulse noise, such as salt-and-pepper noise, is another common type of noise that can severely degrade image quality. Traditional methods for salt-and-pepper noise removal often involve median-based filters or adaptive approaches that exploit local image statistics\cite{Memis2021}. Deep learning methods have also been explored for salt-and-pepper noise removal. SeConvNet introduces a new selective convolutional (SeConv) block that effectively restores salt-and-pepper noise corruption in both grayscale and color images, outperforming traditional methods at high noise densities.\cite{Rafiee2021}

An important architecture in deep learning for image processing is the U-Net. This encoder-decoder architecture with skip connections has proven highly effective for tasks that require precise localization, such as segmentation and image-to-image translation. The U-Net's ability to preserve fine details while capturing global context makes it well-suited for image restoration tasks.\cite{Wu2022}

The proposed SaltNet model builds upon the strengths of existing deep learning approaches for image denoising. It incorporates SeConv blocks from SeConvNet to effectively handle salt-and-pepper noise, while also introducing novel anisotropic diffusion blocks to further enhance denoising capabilities. Additionally, SaltNet leverages Transformer blocks, similar to Restormer, to capture long-range dependencies. SaltNet is terminated by an AutoEncoder U-Net-like architecture to enhance ability to capture fine details and long-range pixel dependencies. By combining these different components, SaltNet aims to achieve competitive denoising performance while maintaining resource efficiency.
